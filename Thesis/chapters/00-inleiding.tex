% 
% Inleiding
% @author Pieter Maene <pieter.maene@student.kuleuven.be>
%

\chapter{Inleiding}
\label{chap:inleiding}

Verkiezingen zijn een essentieel onderdeel van het democratisch proces en spelen dus een zeer belangrijke rol in onze maatschappij. In 2014 vinden in 40 landen nationale verkiezingen plaats. 42\% van de wereldbevolking zal dit jaar hun stem kunnen uitbrengen.\cite{news:economist_2014_ballot_boxes} Ook in Belgi\"e is 2014 een groot verkiezingsjaar: op 25 mei wordt tegelijk gestemd voor het Europees parlement, de Kamer van volksvertegenwoordigers en de verkiezing van de deelstaten (Vlaams of Waals parlement).

\npar In 2012 werd tijdens de presidenti\"ele verkiezingen in Amerika door beide kandidaten ongeveer \'e\'en miljard dollar uitgegeven tijdens hun campagne.\cite{news:nytimes_2012_money_race} In Belgi\"e werd tijdens de federale verkiezingen van 2010 door alle partijen samen 13,7 miljoen euro uitgegeven. Niet alleen de budgetten van de kandidaten zijn zo hoog: de kosten van de organisatie worden voor 25 mei op meer dan 10 miljoen euro geschat.\cite{news:tijd_moeder_aller_verkiezingen_kostprijs}

\npar Gezien het belang van de verkiezingen en de enorme bedragen die ermee gemoeid gaan, is het dus noodzakelijk dat er een betrouwbaar systeem is om het resultaat vast te leggen. Het moet bovendien eenvoudig te gebruiken zijn door de kiezers. Tijdens de Amerikaanse presidentsverkiezingen van 2000 werd in Palm Beach County, Florida een biljet gebruikt dat verwarrend zou zijn. Dit kreeg des te meer aandacht omdat de uitslag erg nipt was.\cite{wiki:united_states_presidential_election_in_florida_2000} In Belgi\"e stemt bijna de helft van de kiezers elektronisch, maar ook hier kan de interface voor problemen zorgen. Tijdens de gemeenteraadsverkiezingen van 2012 kon per ongeluk een voorkeurstem gegeven worden door te lang te duwen op het scherm.\cite{news:maddens_zijn_de_stemcomputers_wel_te_vertrouwen} In veel steden werden voor 25 mei stemcomputers ter beschikking gesteld om op te oefenen.\cite{news:de_redactie_ga_eens_oefenen_op_een_stemcomputer}

\npar Het is dus niet eenvoudig om een gebruiksvriendelijke oplossing te ontwikkelen, noch voor papieren biljetten, noch wanneer digitaal gestemd wordt. Daarnaast heeft de kiezer vandaag geen enkele manier om na te gaan of zijn stem juist meegeteld is en dat het algemene resultaat correct is. Hij moet de instantie de verkiezing organiseert dus volledig vertrouwen. Dit kan opgelost worden door gebruik te maken van een voter-verifiable systeem. Een groot nadeel is echter dat veel van deze systemen een complexe procedure hebben. Helios is een systeem dat het mogelijk maakt om online voter-verifiable verkiezingen te organiseren.

\npar In \ref{chap:literatuurstudie} worden papieren systemen besproken die voter-verifiable zijn. Veel van de technieken die hier gebruikt worden, zullen ook terugkomen in Helios. \ref{chap:helios} tot \ref{chap:interface} behandelen de werking van dit systeem, de procedure die gevolgd moet worden voor het opzetten van een verkiezing en de wijzigingen aan de interface. Het aangepaste systeem werd ook in de praktijk gebruikt om na te gaan hoe bruikbaar het is voor een echte verkiezing (\ref{chap:kringverkiezing}). 

\npar In \ref{chap:sleutels_en_fingerprints} wordt kort gekeken naar methoden om de sleutels en fingerprints te bewaren. Tot slot worden in \ref{chap:web_cryptography_api} de prestaties van de Web Cryptography API ge\"evalueerd.
