% 
% Besluit
% @author Pieter Maene <pieter.maene@student.kuleuven.be>
%

\chapter{Besluit}
\label{chap:besluit}

Om het vertrouwen van kiezers in het resultaat te vergroten, kan een voter-verifiable verkiezingssysteem gebruikt worden. Er zijn verschillende papieren systemen beschikbaar die dit realiseren, maar deze hebben zeer complexe procedures. Hierdoor kunnen ze vaak niet op grote schaal gebruikt worden in de praktijk (\ref{sec:ls:conclusie}). Helios is een voter-verifiable stemsysteem voor online verkiezingen. Dit betekent wel dat het alleen gebruikt kan worden wanneer dwang geen grote bedreiging vormt.

\npar De procedure die gevolgd moet worden om een verkiezing op te zetten in Helios werd besproken in \ref{chap:procedure} De interface werd ook zo aangepast dat deze procedure er veel duidelijker in terug te vinden is (\ref{sec:ui:beheer}). In de praktische tests bleek dat deze nieuwe interface een grote hulp was. Beheerders van de verkiezing konden zonder veel aanwijzingen een verkiezing met threshold encryptie opzetten (\ref{sec:kv:beheer}).

\npar Om Helios te gebruiken in de kringverkiezing van VTK, werden eerst nog enkele aanpassingen doorgevoerd. Vervolgens werd alles grondig getest en op basis van de gegeven feedback nog verder gewijzigd. Vooral de procedure in het stemhokje werd tijdens deze tests als te complex ervaren (\ref{sec:kv:stemhokje}). In \ref{sec:kv:stemdag} werd de stemdag zelf besproken, waarbij geen grote problemen vastgesteld werden.

\npar In \ref{sec:sf:disk} werd gezien dat de geheime sleutels van de trustees bewaard worden in een bestand. Ook wanneer ze deze ergens anders nodig hebben, zullen deze vaak op een USB-stick geplaatst worden. Om de sleutels te beveiligen kan ofwel de partitie ge\"encrypteerd worden, ofwel de bestanden zelf. Als alternatief zouden deze ook in de browser zelf opgeslagen kunnen worden (\ref{sec:sf:web_storage}). Visuele hashes zouden het bewaren en vergelijken van de fingerprints sterk vergemakkelijken (\ref{sec:sf:fingerprints}).

\npar De Web Cryptography API geeft webontwikkelaars eenvoudig toegang tot cryptografische functies. Uit de vergelijkingen met bestaande JavaScript implementaties bleek dat deze ook een grote snelheidsverbetering opleveren. Er wordt wel slechts een beperkt aantal algoritmes ondersteund en de API biedt alleen high-level functies aan, waardoor het zeer moeilijk is om alternatieven te implementeren (\ref{sec:wc:modulaire_exponentiatie}).
