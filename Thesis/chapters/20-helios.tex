% 
% Helios
% @author Pieter Maene <pieter.maene@student.kuleuven.be>
%

\chapter{Helios}
\label{chap:helios}

Het Helios verkiezingssysteem is een open-source project dat geleid wordt door Ben Adida.\cite{adida_helios} Het laat toe om online voter-verifiable verkiezingen te organiseren. Dit systeem werd vorig jaar door Robbert Coeckelbergh uitgebreid met threshold encryptie en gerangschikte verkiezingen.\cite{coeckelbergh_toepassing_en_uitbreiding_van_het_helios_online_verkiezingssysteem} 

\npar In dit hoofdstuk worden de belangrijkste cryptografische technieken uitgelegd die Helios gebruikt (\ref{sec:helios:cryptografische_technieken}). Daarna wordt de functionaliteit van de publieke delen van Helios besproken (\ref{sec:helios:stemhokje} en \ref{sec:helios:ballot_tracking_center}).

\section{Cryptografische technieken}
\label{sec:helios:cryptografische_technieken}

Op het vlak van cryptografische technieken leunt Helios het dichtste aan bij Scratch \& Vote (\ref{sec:ls:scratch_and_vote}). Het belangrijkste verschil is dat ElGamal gebruikt wordt in plaats van Paillier voor de homomorfe encryptie van de stemmen. Tot slot wordt de methode besproken waarop de sleutel verdeeld wordt tussen de trustees (\ref{sec:helios:threshold_encryptie}).

\subsection{ElGamal~\cite{elgamal_elgamal}}
\label{sec:helios:elgamal}

Het ElGamal cryptosystem is een asymmetrisch schema dat gebaseerd is op het Diffie-Hellman protocol. Dit betekent dat een sleutelpaar met zowel een geheime als publieke sleutel nodig is. De publieke sleutel wordt gebruikt om de klaartekst te encrypteren. De decryptie kan alleen uitgevoerd worden met de geheime sleutel.

\npar Er wordt gewerkt in de groep $\mathbb{Z}_p$ waar $g$ de generator is. De geheime sleutel $sk$ wordt willekeurig gekozen binnen $\mathbb{Z}_{p-1}$. De publieke sleutel is dan ${pk} = g^{sk} \mod{p}$. De cijfertekst van een ElGamal encryptie bestaat uit twee delen: $c_1$ (\ref{eq:helios:elgamal_c1}) en $c_2$ (\ref{eq:helios:elgamal_c2}). In deze vergelijkingen is $m$ de klaartekst en $r$ opnieuw een willekeurig getal binnen $\mathbb{Z}_{p-1}$. $c_1$ en $r$ hebben respectievelijk de functie van tijdelijke publieke en private sleutel.\cite{preneel_cryptography_and_network_security}

\begin{align}
  \label{eq:helios:elgamal_c1} 
  c_1 & = g^r \mod{p} \\
  \label{eq:helios:elgamal_c2}
  c_2 & = m \cdot {pk}^r \mod{p}
\end{align}

\npar De cijfertekst kan dan gedecrypteerd worden volgens \ref{eq:helios:elgamal_m}.

\begin{equation}
  \label{eq:helios:elgamal_m}
  m = \frac{c_2}{c_1^{sk}} \mod{p}
\end{equation}

\subsection{Homomorfe encryptie}
\label{sec:helios:homomorfe_encryptie}

Bij homomorfe encryptie kan een specifieke operatie met de cijfertekst uitgevoerd worden. De resulterende cijfertekst is de encryptie van een bepaalde bewerking op de klaarteksten.\cite{wiki:homomorphic_encryption} Zo is het Paillier cryptosystem dat gebruikt wordt in Scratch \& Vote (\ref{sec:ls:scratch_and_vote}) homomorf onder $(\times, +)$. Dit betekent dat een vermenigvuldiging van de cijferteksten resulteert in een optelling van de klaarteksten.

\npar Het homomorfisme $(\times, +)$ kan in een verkiezingssysteem gebruikt worden om effici\"ent de stemmen op te tellen. De berekening van het resultaat kan immers gebeuren aan de hand van de cijferteksten. Er moet nu alleen een decryptie gebeuren om het uiteindelijke resultaat vrij te geven.

\npar Aan de hand van \ref{eq:helios:elgamal_c2} kan gezien worden dat ElGamal standaard homomorf is onder $(\times, \times)$. Zoals hiervoor besproken, is voor een verkiezingssysteem echter het homomorfisme $(\times, +)$ nodig. Dit kan gerealiseerd worden door de klaartekst ook in de exponent te plaatsen (\ref{eq:helios:elgamal_c2_homomorphic}).

\begin{equation}
  \label{eq:helios:elgamal_c2_homomorphic}
  c_2 = g^m \cdot {pk}^r \mod{p}
\end{equation}

\ref{eq:helios:elgamal_homomorphic} geeft het homomorfisme dat zo bekomen wordt.

\begin{equation}
  \label{eq:helios:elgamal_homomorphic}
  \begin{array}{lcl}
    \mathcal{E}(m_1) \cdot \mathcal{E}(m_2) &=& (g^{r_1}, g^{m_1} \cdot {pk}^{r_1}) \cdot (g^{r_2}, g^{m_2} \cdot {pk}^{r_2}) \\
      &=& (g^{r_1 + r_2}, g^{m_1 + m_2} \cdot {pk}^{r_1 + r_2}) \\
      &=& \mathcal{E}(m_1 + m_2)
  \end{array}
\end{equation}

\npar Omwille van het discreet logaritme probleem is het terugvinden van $m$ echter niet meer zo vanzelfsprekend.\cite{menezes_vanstone_oorschot_handbook_of_applied_cryptography} Dit kan alleen gedaan worden door $g^m \mod{p}$ te berekenen voor elke $m$ en vervolgens te zoeken welke hetzelfde is als de klaartekst van de decryptie.

\npar Scratch \& Vote (\ref{sec:ls:scratch_and_vote}) gebruikt multi-counters voor een stemming met meerdere kandidaten. Helios daarentegen encrypteert ieder mogelijk antwoord op een vraag afzonderlijk. Wanneer de optie gekozen wordt, is $m = 1$; anders wordt $m = 0$ gesteld.

\subsection{Threshold encryptie}
\label{sec:helios:threshold_encryptie}

%TODO Duidelijk maken dat 3.7 een enkele stem is?

Oorspronkelijk kon de publieke sleutel voor de verkiezing alleen berekend worden als het product van de afzonderlijke publieke sleutels van de trustees (\ref{eq:helios:elgamal_c2_homomorphic_trustees}). Voor de decryptie moet iedere trustee zijn factor uit de noemer van \ref{eq:helios:elgamal_m_homomorphic_trustees} berekenen. Deze factoren worden in Helios de decryptiefactoren genoemd.

\begin{equation}
  \label{eq:helios:elgamal_c2_homomorphic_trustees}
  PK = \prod_{i=1}^n{{pk}_i} \mod{p}
\end{equation}

\begin{equation}
  \label{eq:helios:elgamal_m_homomorphic_trustees}
  g^m = \frac{c_2}{\prod_{i=1}^n{c_1^{{sk}_i}}} = \frac{c_2}{\prod_{i=1}^n{{df}_i}} \mod{p}
\end{equation}

\npar Een groot probleem hierbij is dat wanneer \'e\'en trustee zijn geheime sleutel verliest, het resultaat niet meer gedecrypteerd kan worden. Daarom werd threshold encryptie toegevoegd door Robbert Coeckelbergh.\cite{coeckelbergh_toepassing_en_uitbreiding_van_het_helios_online_verkiezingssysteem} Er kan nu een threshold schema gedefinieerd worden zodat slechts $k$ van de $n$ trustees hun decryptiefactor moeten berekenen.

\subsubsection{Secret Sharing}

De methode die ge\"implementeerd werd is gebaseerd op Shamir's secret sharing.\cite{shamir_how_to_share_a_secret} Iedere trustee genereert eerst een veelterm van graad $k - 1$. Vervolgens stuurt hij elke trustee (ook zichzelf) een zogeheten \textit{share} van deze rechte. Deze share is de waarde van de rechte voor een punt $x$. Deze $x$-co\"ordinaat moet door iedereen gekend zijn, omdat het vereist is dat de shares die een trustee ontvangt van de anderen voor dezelfde waarde werden aangemaakt. Daarom wordt hiervoor binnen Helios het database ID van de trustee gebruikt. Dit is een uniek natuurlijk getal dat wordt opgehoogd telkens een nieuwe trustee aangemaakt wordt.

\npar Vervolgens moet de trustee de $n$ shares die hij zo ontvangt, optellen. Zo bekomt hij de $y$-co\"ordinaat die hoort bij zijn ID op een nieuwe veelterm, die de som is van de $n$ veeltermen die door de trustees gegenereerd werden. Deze waarde kan hij nu gebruiken als zijn geheime sleutel ${sk}$. Omdat ElGamal gebruikt wordt als encryptieschema, wordt zijn publieke sleutel ${pk} = g^{sk} \mod{p}$.

\npar Als geheime sleutel voor de verkiezing wordt nu de waarde voor $x = 0$ op de gemeenschappelijke veelterm genomen. Deze veelterm kan door Lagrange-interpolatie gereconstrueerd worden uit $k$ punten. Hiervoor worden de eerste $k$ trustees gebruikt (dat zijn deze met het laagste ID). Omdat alleen de publieke sleutels van de trustees beschikbaar zijn, wordt echter onmiddellijk de publieke sleutel voor de verkiezing berekend (\ref{eq:helios:threshold_encryption_public_key}).

\begin{align}
  \label{eq:helios:threshold_encryption_lagrange}
  \lambda_i(x) & = \prod_{j=1, j\not=i}^k{\frac{x - x_j}{x_i - x_j}} \\
  \label{eq:helios:threshold_encryption_polynomial}
  V(x) & = \sum_{i=1}^k{{sk}_i\lambda_i(x)}
\end{align}

\begin{align}
  \label{eq:helios:threshold_encryption_secret_key}
  SK & = V(0) = \sum_{i=1}^k{{sk}_i\lambda_i(0)} \\
  \label{eq:helios:threshold_encryption_public_key}
  PK & = g^{X} = \prod_{i=1}^k{{pk}_i^{\lambda_i(0)}} \mod{p}
\end{align}

\npar Om het resultaat te decrypteren moet de geheime sleutel voor de verkiezing gebruikt worden (\ref{eq:helios:threshold_encryption_secret_key}). Het grote voordeel van threshold encryptie is dat het hier niet belangrijk is van welke $k$ trustees de decryptiefactoren en bijhorende Lagrange-interpolatie gebruikt worden.

\begin{equation}
  \label{eq:helios:threshold_encryption_m}
  g^m = \frac{c_2}{\prod_{i=1}^k{c_1^{{sk}_i\lambda_i(0)}}} = \frac{c_2}{\prod_{i=1}^k{{df}_i^{\lambda_i(0)}}} \mod{p}
\end{equation}

%TODO ZK Proofs?

\subsubsection{Communicatiesleutels}
\label{sec:helios:communicatiesleutels}

Voordat iedere trustee zijn gegenereerde shares doorstuurt naar de andere trustees, worden deze ge\"encrypteerd en getekend. Hiervoor wordt respectievelijk de publieke sleutel voor encryptie en voor tekenen van de andere trustee gebruikt. Dit geeft aanleiding tot twee nieuwe sleutelparen die de communicatiesleutels genoemd worden.

\section{Stemhokje}
\label{sec:helios:stemhokje}



\section{Ballot Tracking Center}
\label{sec:helios:ballot_tracking_center}



\section{Controleapplicatie}
\label{sec:helios:controleapplicatie}


