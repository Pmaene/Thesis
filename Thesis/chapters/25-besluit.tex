% 
% Besluit
% @author Pieter Maene <pieter.maene@student.kuleuven.be>
%

\chapter{Besluit}
\label{chap:besluit}

Om het vertrouwen van kiezers in het resultaat te vergroten, kan een voter-verifiable verkiezingssysteem gebruikt worden. Er zijn verschillende papieren systemen beschikbaar die dit realiseren, maar deze hebben zeer complexe procedures. Hierdoor kunnen ze in de praktijk vaak niet op grote schaal gebruikt worden. Helios is een voter-verifiable stemsysteem voor online verkiezingen. Dit betekent wel dat het alleen gebruikt kan worden wanneer dwang geen grote bedreiging vormt.

\npar De procedure die gevolgd moet worden om een verkiezing op te zetten in de aangepaste versie van Helios werd besproken. De interface werd ook zo aangepast dat deze procedure er veel duidelijker in terug te vinden is. In de praktische tests bleek dat deze nieuwe interface een grote hulp was. Beheerders van de verkiezing konden zonder veel aanwijzingen een verkiezing met threshold encryptie opzetten.

\npar Om Helios te gebruiken in de kringverkiezing van VTK, werden eerst nog enkele aanpassingen doorgevoerd. Vervolgens werd alles grondig getest en op basis van de gegeven feedback nog verder gewijzigd. Vooral de procedure in het stemhokje werd tijdens deze tests als te complex ervaren. Het gewijzigde systeem werd tenslotte gebruikt voor een echte verkiezing, waarbij geen grote problemen vastgesteld werden. Er kwamen ook geen klachten over het systeem.

\npar Tot slot werden twee mogelijke toekomstige uitbreidingen besproken. Ten eerste werden aanbevelingen gedaan voor het bewaren van geheime sleutels en fingerprints. De geheime sleutels van de trustees worden bewaard in een bestand. Om de sleutels te beveiligen kan ofwel de partitie ge\"encrypteerd worden, ofwel de bestanden zelf. Als alternatief zouden deze ook in de browser zelf opgeslagen kunnen worden. Visuele hashes zouden het bewaren en vergelijken van de fingerprints sterk vergemakkelijken.

\npar Ten tweede werden de prestaties van de Web Cryptography API onderzocht. Deze geeft web ontwikkelaars toegang tot cryptografische functies die in de browser ingebouwd zijn. Uit de vergelijkingen met bestaande JavaScript implementaties bleek dat deze ook een grote snelheidsverbetering opleveren. Er wordt wel slechts een beperkt aantal algoritmes ondersteund en de API biedt alleen high-level functies aan, waardoor het zeer moeilijk is om alternatieven te implementeren.
