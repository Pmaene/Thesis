%
% Short Abstract
% @author Pieter Maene <pieter.maene@student.kuleuven.be>
%

Een voter-verifiable stemsysteem geeft de kiezer de mogelijkheid om na te gaan of zijn eigen stem correct geregistreerd is en of het resultaat correct is. Het is mogelijk om een dergelijk systeem met papieren stembiljetten te implementeren, zowel met als zonder cryptografische technieken. Helios is een voter-verifiable stemsysteem voor online verkiezingen. De procedure die bij deze systemen gevolgd moet worden, is echter vaak zeer complex. In deze thesis wordt de procedure gegeven die in Helios gevolgd moet worden om een verkiezing op te zetten. De interface van het systeem werd ook herwerkt om de beheerder beter te ondersteunen. Ook de applicatie die gebruikt wordt door de kiezers werd vereenvoudigd. Na het systeem uitgebreid getest te hebben, werd het in de praktijk gebruikt voor een re\"ele verkiezing waarbij ongeveer 750 mensen hun stem uitbrachten. Daarnaast wordt besproken hoe de trustees hun geheime sleutels kunnen bewaren. Er wordt ook onderzocht of de tekstuele fingerprints niet op een betere manier weergegeven kunnen worden. Tot slot worden de prestaties van de Web Cryptography API vergeleken met deze van bestaande implementaties in JavaScript. Deze nieuwe specificatie geeft ontwikkelaars toegang tot cryptografische functies die in de browser ingebouwd zijn.