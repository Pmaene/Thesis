% 
% Systemen voor Online Verkiezingen
% @author Pieter Maene <pieter.maene@student.kuleuven.be>
%

\documentclass[master=elt,masteroption=im,inputenc=utf8]{kulemt}
\setup{title={Systemen voor Online Verkiezingen},
  subtitle={},
  author={Pieter Maene},
  promotor={Bart Preneel},
  assessor={Vincent Rijmen \and Luc Van Eycken},
  assistant={Jens Hermans \and Frederik Vercauteren}
}

\setup{filingcard,
  translatedtitle={Systems for Online Elections},
  udc=621.3,
  shortabstract={}
}

% Kies de fonts voor de gewone tekst, bv. Latin Modern
\setup{font=lm}

% Hier kun je dan nog andere pakketten laden of eigen definities voorzien
\usepackage{amsmath,amssymb,mathtools}
\usepackage{color}
\usepackage{enumitem}
\usepackage{eurosym}
\usepackage{float}
\usepackage{kulirx}
\usepackage{multirow,bigstrut}
\usepackage{pdfpages}

% De optie colorlinks mag verwijderd worden voor de af te drukken versie.
\usepackage[pdfusetitle,colorlinks,plainpages=false]{hyperref}

% Verwijder de "%" op de volgende lijn als je de kaft wil afdrukken
%\setup{coverpageonly}

% Customisations
\graphicspath{{./images/}}

\setlength{\parindent}{0pt}
\setlist[description]{topsep=0.5em, itemsep=0.25em, leftmargin=2em, labelindent=2em}

\headstyles{kulirman}

% Alias
%%
% Alias
% @author Pieter Maene <pieter.maene@student.kuleuven.be>
%

\renewcommand{\ref}[1]{\mbox{\autoref{#1}}}

\renewcommand{\chapterautorefname}{Chapter}
\renewcommand{\equationautorefname}{Eq.}
\renewcommand{\figureautorefname}{Fig.}
\renewcommand{\footnoteautorefname}{Footnote}
\renewcommand{\itemautorefname}{Item}
\renewcommand{\sectionautorefname}{Section}
\renewcommand{\subsectionautorefname}{Section}
\renewcommand{\subsubsectionautorefname}{Section}
\renewcommand{\paragraphautorefname}{Paragraph}
\renewcommand{\subparagraphautorefname}{Paragraph}
\renewcommand{\pageautorefname}{Page}
\renewcommand{\tableautorefname}{Table}

\newcommand{\cplusplus}{C\texttt{++}\ }

% Document
\begin{document}

  % Alias
  %
% Alias
% @author Pieter Maene <pieter.maene@student.kuleuven.be>
%

\renewcommand{\ref}[1]{\mbox{\autoref{#1}}}

\renewcommand{\chapterautorefname}{Chapter}
\renewcommand{\equationautorefname}{Eq.}
\renewcommand{\figureautorefname}{Fig.}
\renewcommand{\footnoteautorefname}{Footnote}
\renewcommand{\itemautorefname}{Item}
\renewcommand{\sectionautorefname}{Section}
\renewcommand{\subsectionautorefname}{Section}
\renewcommand{\subsubsectionautorefname}{Section}
\renewcommand{\paragraphautorefname}{Paragraph}
\renewcommand{\subparagraphautorefname}{Paragraph}
\renewcommand{\pageautorefname}{Page}
\renewcommand{\tableautorefname}{Table}

\newcommand{\cplusplus}{C\texttt{++}\ }

  \tableofcontents*

  \renewcommand{\abstractname}{Inleiding}
  \begin{abstract}
    %TODO Inleiding
  \end{abstract}

  \mainmatter

  % Chapters
  % 
% Literatuurstudie
% @author Pieter Maene <pieter.maene@student.kuleuven.be>
%

\chapter{Literatuurstudie}
\label{chap:literatuurstudie}

Deze thesis handelt over methoden voor online verkiezingen. Een interessant verwant probleem zijn systemen die gebaseerd zijn op papier. Eerst wordt een kort overzicht van de geschiedenis van stemsystemen gegeven. Vervolgens worden de belangrijkste vereisten bekeken waaraan deze systemen moeten voldoen. Belangrijk hierbij is de definitie van een voter verifiable systeem. Tot slot onderzoeken we zowel systemen die geen gebruik maken van cryptografische methoden als deze die daar wel op steunen.

\section{Geschiedenis}

Onze samenleving heeft een rijke geschiedenis van stemprocedures, die teruggaat tot Athene in het oude Griekenland. Hier bracht men een negatieve stem uit op een potscherf. In deze paragraaf bekijken we kort de meest relevante voor de manier waarop we vandaag te werk gaan.

\npar Sinds de uitvinding van het geheime stembiljet in 1858 in Australi\"e is er eigenlijk niet meer zoveel veranderd. In dit systeem worden de biljetten op voorhand gedrukt door de staat en veilig bewaard tot op de stemdag. Elke stemgerechtigde krijgt op de stemdag een biljet waarna hij zijn stem uitbrengt in een stemhokje. Het grootste voordeel van deze methode ligt in het feit dat elke stem geheim is.

\npar Deze stemmethode maakte het daarnaast ook mogelijk om mechanische (en later elektrische) machines te gebruiken. Mechanische systemen werden gebruikt in grotere gemeenschappen en waren gebaseerd op hendels en mechanische tellers. De eerste van deze machines werden in 1892 ingevoerd in New York. Rond 1960 werden de eerste elektrische machines ingevoerd. Deze maakten gebruiken van optische scans. Bij deze systemen moet de stem meestal op een specifieke manier aangegeven worden, bijvoorbeeld door het inkleuren van bolletjes.

\npar Sinds 2000 worden Direct Recording by Electronics (DRE) machines steeds vaker gebruikt. Hierop draait speciale stemsoftware, die de keuze van de kiezer digitaal vastlegt. Deze machines maken het stemproces aanzienlijk eenvoudiger. Het grootste probleem is dat er geen enkele bevestiging aan de kiezer gegeven wordt en dat hij deze machines dus volledig moet vertrouwen.

\npar Het gebrek aan controle door de kiezer bij DRE vormde de aanleiding voor het ontwerpen van Voter-Verified Paper Audit Trails (VVPAT) machines. Hierbij toont de machine de kiezer een afgeschermde afdruk van zijn stem, waarna hij deze kan accepteren of weigeren. Op die manier kan de kiezer verifi\"eren dat zijn stem correct is. In principe zouden bij een hertelling dan ook deze papieren tickets en niet de digitale data gebruikt moeten worden.

\section{Vereisten}

Bij het ontwerpen van een stemsysteem zijn er twee tegenstrijdige doelen. Enerzijds moet het mogelijk zijn dat de kiezer thuis kan controleren of zijn stem juist meegeteld is. Anderzijds mag diezelfde persoon niet kunnen bewijzen voor wie hij precies gestemd heeft. Wanneer hij dit wel kan, zou hij zijn stem kunnen verkopen aan iemand die de verkiezing wil beïnvloeden.

\npar Hoewel het vaak zeer moeilijk is om een grote verkiezing doorslaggevend te wijzigen, wordt stemfraude toch regelmatig geconstateerd.\cite{adida_advances_in_cryptographic_voting_systems} E\'en van de grote moeilijkheden is dat zowel kiezers als bijzitters corrupt kunnen zijn. Er kan dus van geen enkele deelnemer verwacht worden dat hij eerlijk is. Bovendien is het zeer moeilijk om fouten te ontdekken in de huidige systemen, daar zeer veel informatie vernietigd wordt om de geheime stemming te garanderen.

\subsection{Vertrouwen}
\label{sec:ls:vertrouwen}

De huidige manier van stemmen vereist dat de kiezer zeer veel vertrouwen legt in het gebruikte systeem. Zoals verder besproken wordt (\ref{sec:ls:end_to_end_verifiability}), zijn er nieuwe ontwerpen waarbij de kiezer kan controleren of zijn stem correct meegeteld is. Deze systemen steunen vaak op moeilijke cryptografische technieken, die heel wat achtergrondkennis vragen om ze te begrijpen.

\npar Een vereiste voor om het even welk stemsysteem is dat het vertrouwd wordt door een gemiddelde kiezer, de bijzitters en de publieke opinie en media. Opdat deze mensen een dergelijk systeem zouden vertrouwen, moeten de experts die het systeem goedkeuren dit op een eenvoudige manier aan hen kunnen uitleggen.\cite{randell_ryan_voting_technologies_and_trust} Daarom zullen eenvoudige systemen die geen gebruik maken van cryptografie waarschijnlijk sneller aanvaard worden door een breed publiek.

\subsection{End-to-End Verifiability}
\label{sec:ls:end_to_end_verifiability}

In een end-to-end verifiably voting systeem wordt niet nagegaan of de code van de stemmachines volledig correct is. In plaats daarvan wordt wiskundig bewezen dat de output correct is. Op die manier kan de moeilijke en vaak ondoorzichtige fysische chain-of-custody vermeden worden. Dit betekent ook dat iemand niet langer speciale toegang moet hebben om de resultaten te controleren. Om het even wie kan nagaan of de bewijzen correct zijn.

\npar Door gebruik te maken van cryptografische technieken kan een dergelijk systeem veilig ge\"implementeerd worden. Door de stemmen te encrypteren wordt verzekerd dat ze geheim blijven. Door het geven van een zero-knowledge bewijs kan aangetoond worden dat de stemmen correct geteld zijn.

\section{Systemen zonder cryptografie}
\label{sec:ls:systemen_zonder_cryptografie}

In deze sectie worden enkele systemen besproken waarin geen gebruik gemaakt wordt van cryptografie. Open Counting (\ref{sec:ls:open_counting}) is een techniek waarbij alleen de telfase aangepast wordt. Floating receipts (\ref{sec:ls:floating_receipts}) kunnen de veiligheid van elk papieren stemsysteem sterk verbeteren. ThreeBallot (\ref{sec:ls:scratch_card}) en Scratch-Card zijn beiden voter-verifiable systemen die gebruik maken van papieren tickets. Twin (\ref{sec:ls:twin}) en VAV (\ref{sec:ls:vav}) bouwen verder op respectievelijk floating receipts en ThreeBallot. Vooral ThreeBallot wordt in detail besproken omdat de belangrijkste concepten van papier-gebaseerde voter-verifiable verkiezingen hierin aan bod komen.

\subsection{Open Counting}
\label{sec:ls:open_counting}

Open counting vertrekt van de systemen zoals we ze vandaag kennen, maar de stemmen worden op een nieuwe manier geteld.\cite{adi_schuler_frohlich_open_counting} Het stembiljet is aangepast om eenvoudig optisch geteld te worden. De stemmen worden nog steeds geteld door ambtenaren. Elke stem wordt op een scherm getoond aan verschillende telstations, elk met hun eigen hardware die het getoonde biljet filmt en analyseert. Ieder station geeft op regelmatige tijdstippen zijn huidig totaal en wanneer er onenigheid is, wordt het gedisputeerde biljet gezocht en het probleem opgelost.

\npar Tijdens het tellen geeft ieder station ook een hash van hun opgenomen video. Hiervoor wordt een veilige hash-functie gebruikt. Deze hashes kunnen dan later gebruikt worden tijdens een geautomatiseerde audit om te controleren of er niet geknoeid is met de beelden. Dit proces is publiek en dus kan iedereen zijn eigen hardware meebrengen en de telling zelf uitvoeren. Het systeem wordt zo ontworpen dat een eenvoudige camera en computer volstaan. Ook deze waarnemers kunnen een hash van hun video publiceren om geloofwaardiger over te komen.

\npar De verschillende telstations controleren continu elkaar en ook de waarnemers kunnen achteraf onregelmatigheden melden. Omdat het systeem snel werkt, kan de telling in het stembureau zelf gehouden worden. Op die manier kunnen alle belanghebbenden aanwezig zijn en kan het transport van de ongetelde biljetten vermeden worden. Het transparante karakter en het gebruik van eenvoudige hardware kunnen het vertrouwen van kiezers in het systeem sterk verbeteren.

\subsection{Floating Receipts}
\label{sec:ls:floating_receipts}

Floating receipts\cite{rivest_smith_three_voting_protocols} zijn een waardevolle toevoeging voor elk papieren telsysteem. Een doos met stembiljetten wordt aan de uitgang van het stembureau geplaatst. De kiezer maakt bij het buitengaan een kopie van een ticket dat hij hieruit trekt, vooraleer hij zijn eigen erbij legt. Hij neemt dus een random ticket mee dat niet het zijne is, maar hij kan dit ticket toch later gebruiken om te controleren of de stemprocedure correct verlopen is. Omdat de doos initieel leeg is, krijgen de eerste $T$ kiezers geen ticket mee naar huis. Hierbij is $T$ een constante die veel kleiner is dan het aantal kiezers, maar voldoende groot zodat $1/T$ een kleine kans voorstelt.

\npar Niemand weet dus met grote waarschijnlijkheid van wie hij het biljet gekopieerd heeft. Een deel van de originele tickets wordt gekopieerd, maar niemand kan achterhalen welke precies. Omdat de aanvaller geen betrouwbare methode heeft om alle kopie\"en van een ticket te bemachtigen, is het systeem bestendig tegen het vervangen van biljetten of het verkopen van een stem. Een nadeel is dat kiezer niet langer zijn eigen ticket heeft en dus misschien minder gemotiveerd is om te controleren of dit correct meegeteld is. Er wordt echter verondersteld dat een groot aantal kiezers dat toch nog steeds zal doen.

\subsubsection{Short Ballot Assumption}

Bij de \textit{Short Ballot Assumption} (SBA) is het belangrijk dat het aantal kandidaten op het biljet beperkt blijft. Er is dan een groot aantal verschillende manieren zijn om de verschillende delen met elkaar te combineren. Een aanvaller kan daar dan onvoldoende informatie uithalen om een biljet aan een kiezer te koppelen. Wanneer het aantal mogelijkheden om een biljet in te vullen toeneemt, wordt de kans kleiner dat iemand anders zijn biljet op identiek dezelfde manier invult.\cite{cichon_kutylowski_weglorz_short_ballot_assumption}

\subsubsection{Twin}
\label{sec:ls:twin}

Twin is een voter-verifiable uitbreiding van het klassieke systeem door het gebruik van floating receipts.\cite{rivest_smith_three_voting_protocols} Een traditioneel stembiljet wordt door elke kiezer individueel ingevuld. Onderaan het stembiljet wordt een ID geplaatst, maar dit wordt verborgen door een kraslaag. Na het invullen wordt het biljet gecontroleerd door een machine die deze laag eraf haalt en het biljet in een doos deponeert. Alle kiezers na de $T^{de}$ krijgen een ticket van een willekeurig biljet mee naar huis. Wanneer de stemming afgelopen is, worden alle verzamelde biljetten gepubliceerd op een bulletin board.

\npar Twin is een heel eenvoudig systeem, zonder ingewikkelde wiskunde of specifieke regels voor het correct invullen van het stembiljet. Aangezien het ticket een kopie is van het biljet van iemand anders, kan een kiezer zijn stem niet verkopen. Daarnaast is het zowel voor de tellers als een aanvaller moeilijk om het resultaat ingrijpend te veranderen zonder gedetecteerd te worden. Er wordt immers verondersteld dat een groot aantal kiezers nagaat of zijn ticket correct op het bulletin board staan.

\subsection{ThreeBallot}
\label{sec:ls:threeballot}

ThreeBallot\cite{rivest_threeballot} is een E2E stemsysteem dat ontworpen werd door Ronald Rivest in 2006. Het systeem maakt gebruik van een speciaal stembiljet dat bestaat uit drie identieke delen. Zoals we zullen zien, wordt er gestemd door het invullen van rijen en het deponeren van kolommen. Door alle stembiljetten samen met een lijst van de kiezers op een publieke website (het bulletin-board) te plaatsen, wordt het systeem end-to-end verifiable. De kiezer kan controleren dat zijn stem doorgegeven werd zoals hij bedoelde en dat ze ook correct meegeteld werd.

\subsubsection{Multi-Ballot}
\label{sec:ls:multi-ballot}

De drie delen waaruit het stembiljet bestaat, kunnen ofwel op één blad geprint worden ofwel op meerdere met perforaties ertussen. De drie delen zijn identiek, op een unieke identifier na. De drie IDs op een multi-ballot hebben bovendien geen enkel verband met elkaar of met deze op de andere. In \ref{fig:ls:threeballot} wordt een voorbeeld van een multi-ballot getoond.

\begin{figure}[H]
	\center{\includegraphics[width=0.5\linewidth]{ls/threeballot.png}}
	\caption{Multi-Ballot\cite{rivest_threeballot}}
	\label{fig:ls:threeballot}
\end{figure}

\npar Bij het invullen van het multi-ballot gelden de volgende regels. Elke rij van drie bolletjes komt overeen met \'e\'en kandidaat. Om voor een kandidaat te stemmen, moet de kiezer exact twee bolletjes inkleuren. Om tegen te stemmen, moet er \'e\'en bolletje aangeduid worden. In elke rij moet ten minste \'e\'en bolletje ingevuld zijn, anders is het biljet ongeldig. Ook wanneer er in een bepaalde rij drie bolletjes ingekleurd zijn, is het biljet ongeldig. Hoe de verschillende delen ingevuld zijn, maakt hierbij niet uit: op het ene kunnen alle bolletjes aangeduid zijn, terwijl het andere helemaal leeg is.

\npar Omdat het belangrijk is voor de telling dat de kiezer deze regels juist volgt, moet hij na het invullen van het biljet dit invoeren in een controlemachine. Deze verifieert dat alles correct ingevuld is. Wanneer het biljet niet correct ingevuld is, dan geeft de machine aan waar de kiezer een fout gemaakt heeft. Indien alles wel juist is aangeduid, dan wordt een rode streep geprint op het biljet waarna hij de aparte delen indient. Deze controlemachine mag geen enkele opname maken van de ingevoerde biljetten.

\npar Voordat hij de drie aparte biljetten afgeeft, moet hij er willekeurig \'e\'en uitkiezen waarvan hij een kopie meekrijgt als ticket. Het is het veiligste om dit te implementeren in de controlemachine: nadat het biljet goedgekeurd is, kan op een knop geduwd worden om aan te geven welk deel gekopieerd moet worden. Door de manier waarop het biljet ingevuld is, geeft het ticket geen informatie over hoe de kiezer gestemd heeft.

\subsubsection{Tellen van de stemmen}

Wanneer de verkiezing afgelopen is, worden alle biljetten gescand en de gegevens op het bulletin board gepost. Merk op dat het eigenlijke biljet niet online gezet wordt, omdat de kiezer hier iets op zou kunnen schrijven. Ook een lijst van iedereen die deelgenomen heeft aan de stemming wordt geupload. Een kiezer kan nu nagaan of zijn ticket ook op het bulletin board staat. Wanneer deze ontbreekt, kan hij naar een ambtenaar gaan. Deze gaat na of het ticket echt is en kan dan een hertelling van de stemming vragen.

\npar Omdat alle stemmen op het bulletin board staan, kan iedereen zelf de telling uitvoeren en verifi\"eren of het resultaat correct is. De stemmen kunnen zoals anders geteld worden, zij het met een kleine aanpassing. Omdat er twee bolletjes gekleurd zijn bij een voorstem en maar \'e\'en bij een tegenstem, is het resultaat vermeerderd met het aantal kiezers. Stel dat er $n$ kiezers zijn, waarvan er $a$ voor \textit{Alex Jones} gestemd hebben. Op het einde van het telproces zal het totaal voor deze kandidaat dan gelijk zijn aan $n+a$. De eigenlijk aantallen kunnen dus eenvoudig teruggevonden worden door het aantal kiezers af te trekken van het getelde aantal bolletjes.

\subsubsection{Integriteit}
\label{sec:ls:integriteit}

Door het toevoegen van een ticket en het bulletin board kan de kiezer nagaan dat zijn stembiljet gepubliceerd is en dat het totale aantal geregistreerde biljetten klopt. Deze nieuwe controles zullen ons toelaten om verschillende vormen van fraude eenvoudig te detecteren. Het is ook belangrijk te kijken of zij zelf geen nieuwe zwakheden introduceren.

\npar Het toevoegen van nieuwe stemmen is onmogelijk zonder ook de lijst met kiezers aan te passen. Dit laatste zou normaal opgemerkt moeten worden (bv. door de familie). Daarnaast kunnen er ook geen stemmen aangepast of verwijderd worden zonder dat er mogelijk een kiezer komt klagen dat zijn stem niet correct online staat. Een aanvaller zou wel kunnen proberen om enkele stemmen te veranderen, in de hoop dat dit niet gezien wordt. Grootschalige fraude wordt op deze manier echter onmogelijk.

\npar Een belangrijke doelstelling bij het ontwerp van ThreeBallot was het onmogelijk maken van het verkopen van stemmen en gedwongen stemmen. Zoals eerder besproken, is de volgorde van de bolletjes op de drie deelbiljetten willekeurig. Bovendien krijgt de kiezer maar \'e\'en van de drie delen mee, zodat hij onmogelijk kan aantonen op wie hij effectief gestemd heeft. Een koper kan hem bijvoorbeeld vragen om een ticket met een bepaald patroon mee te brengen, maar op de andere twee delen kan hij nog steeds zijn zin doen.

\npar Bij de \textbf{Three-Pattern} aanval, vraagt de verkoper aan de kiezer om alle drie de delen in een bepaald patroon in te vullen. Wanneer hij dit patroon dan niet terugvindt op het bulletin board, dan wordt de kiezer niet betaald. Er zijn oplossing voor dit probleem, maar deze maken het systeem in de praktijk veel minder bruikbaar. Een mogelijk oplossing is het gebruik van een DRE/EPB machine. Deze print zelf de deelbiljetten in een willekeurig patroon nadat de kiezer zijn keuze gemaakt heeft op een scherm.

\npar Bemerk tot slot dat de controlemachine die de geldigheid van de tickets nagaat zeer goed getest moet worden. Waneer deze aangepast zou worden, kan ze bijvoorbeeld kiezers toelaten om voor een bepaalde kandidaat drie bolletjes te kleuren en voor een andere geen. Zo zouden die stemmen veel meer gewicht krijgen dan deze die zich wel aan de regels houden. Het is bovendien onmogelijk om dergelijke ongeldige patronen achteraf terug te vinden omdat de verschillende biljetten los van elkaar worden ingediend.

\npar Tot slot zou een aanvaller kunnen betalen voor het ticket van de kiezer. Zo kan deze de correctheid van zijn stem niet meer nagaan. De aanvaller zou dan in theorie het biljet kunnen aanpassen dat op het bulletin board geplaatst werd. De kiezers moeten dus aangemoedigd worden om hun ticket niet af te geven. Deze aanval kan eenvoudig tegengegaan worden wanneer de kiezer zonder medeweten van de aanvaller een kopie maakt van het ticket. Voor een digitaal getekend ticket (bv. met een barcode) volstaat dit namelijk ook om klacht neer te leggen.

\subsubsection{Stemgeheim}

Zoals eerder aangehaald, bevat het ticket zelf geen informatie over hoe de kiezer gestemd heeft. Het is dus onmogelijk dat hij per ongeluk zijn stem bekend maakt aan iemand door die persoon het ticket te tonen. Het mag echter ook niet mogelijk zijn om de drie deelbiljetten aan elkaar te linken. Het ID op het ticket zou anders gebruikt kunnen worden om uit te zoeken welk tripel het zijne is. Een vereiste voor het systeem is ook dat niemand vooraf weet welke drie deelbiljetten zullen samenhoren. Een mogelijk oplossing hiervoor is om de delen apart te houden en er willekeurig drie te laten trekken door de kiezer.

\npar De kiezer mag zijn eigen multi-ballot niet kunnen reconstrueren van de biljetten die op het bulletin board gepost worden. Dit kan opgelost worden door het ID te printen in de vorm van een 1D of 2D barcode, wat moeilijk te onthouden is. Het is ook verstandig om de kiezer geen vrije toegang te geven tot een kopieermachine bij het maken van het ticket. Hij zou deze kunnen gebruiken om een kopie te maken van alle drie de deelbiljetten. Dit is de reden dat het ticket best aangemaakt wordt door de controlemachine, wat de eenvoudigste oplossing is om dit tegen te gaan.

\npar Het moet ook onmogelijk zijn voor de kiezer om zijn stem nog te wijzigen nadat ze aanvaard is door de controlemachine. Anders zou hij ze kunnen aanpassen zodat ze ongeldig wordt en zwaarder weegt dan die van anderen. Een eerste oplossing is om hem geen fysische toegang meer te geven tot de biljetten nadat ze gecontroleerd zijn. Een tweede is om samen met de rode streep (\ref{sec:ls:multi-ballot}) een checksum op het biljet te printen die moeilijk veranderd kan worden door de kiezer.

\npar Tot slot moet er opgepast worden dat een \textbf{Reconstructie} aanval niet mogelijk is. Hierbij haalt een aanvaller alle mogelijke geldige multi-ballots uit de biljetten die op het bulletin board geplaatst werden. Samen met het ticket van de kiezer zou hij dan in sommige gevallen kunnen achterhalen hoe deze gestemd heeft. Dit probleem kan elegant opgelost worden door de kiezer te verplichten een ander ticket dan het zijne mee naar huis te nemen. De aanvaller kan immers alleen aan de hand van het ticket achterhalen van wie het gereconstrueerd tripel is. Om de integriteit van de stemming te kunnen controleren, heeft de kiezer alleen het ticket van een geldig biljet nodig. Een mogelijke manier om dit te implementeren is door gebruik te maken van floating receipts (\ref{sec:ls:floating_receipts}). Deze wordt niet expliciet vermeld in de paper van Rivest, maar hij bespreekt wel gelijkaardige methoden.\cite{rivest_threeballot}

\subsubsection{Bruikbaarheid}

Het ThreeBallot systeem is veel complexer dan de manier waarop nu gestemd wordt. De belangrijkste manier om ervoor te zorgen dat het systeem goed werkt is dan ook het opleiden van de kiezer. Het is ook moeilijker om het biljet te corrigeren wanneer er een fout gemaakt wordt: meestal is de enige optie om opnieuw te beginnen met een blanco biljet. Het gebruik van DRE machines, die zelf de nodige bolletjes willekeurig op het biljet zetten, zou het stemmen ook sterk vereenvoudigen. De kiezer moet dan wel controleren dat het geprinte biljet correct is. In \ref{sec:ls:integriteit} werd reeds aangegeven dat dit ook de ThreePattern aanval onmogelijk maakt.

\npar Tot slot merken we nog op dat het tellen van de stemmen wel meer werk vraagt, aangezien er drie keer zoveel biljetten geteld moeten worden. ThreeBallot vergroot het vertrouwen van de kiezer in de integriteit van de verkiezing, ten koste van een iets moeilijker stemproces en meer werk bij het tellen.

\subsubsection{Vote/Anti-Vote/Vote}
\label{sec:ls:vav}

Vote/Anti-Vote/Vote maakt net zoals ThreeBallot gebruik van drie deelbiljetten. Elk biljet (\ref{fig:ls:vav}) wordt vooraf als Vote of Anti-Vote gemarkeerd. De kiezer moet twee Votes en \'e\'en Anti-Vote zodanig invullen dat \'e\'en van de Votes identiek hetzelfde is als de Anti-Vote. Voordat het tellen begint moeten deze identieke biljetten eerst eruit gehaald worden. Net zoals bij ThreeBallot neemt de kiezer een kopie van \'e\'en van de drie delen mee naar huis.

\npar Het is belangrijk dat het Vote-biljet niet teveel informatie bevat. Anders wordt de kans te groot dat het uniek ingevuld wordt. We veronderstellen dus opnieuw dat de SBA geldig is. Voor verkiezingen waarbij maar \'e\'en winnaar kan zijn, kan beter VAV gebruikt worden dan ThreeBallot. Bij zo'n verkiezing met $C$ kandidaten zijn er bij VAV $C$ mogelijke patronen. Bij ThreeBallot zijn er $2^C$ combinaties, wat vaak te groot is voor de SBA.

\begin{figure}[H]
	\center{\includegraphics[width=0.5\linewidth]{ls/vav.png}}
	\caption{VAV biljet\cite{rivest_smith_three_voting_protocols}}
	\label{fig:ls:vav}
\end{figure}

\subsection{Scratch-Card}
\label{sec:ls:scratch_card}

Scratch-Card\cite{randell_ryan_voting_technologies_and_trust} maakt gebruik van een speciaal biljet dat makkelijk in twee gedeeld kan worden (\ref{fig:ls:scratch-card}). Belangrijk is dat de kandidaten op elk biljet in een willekeurige volgorde moeten staan. Een kiezer moet een willekeurig biljet trekken. Na het stemmen moet de kiezer het linkerdeel vernietigen. Hij kan een kopie van het rechterdeel als ticket mee naar huis nemen.

\begin{figure}[H]
	\center{\includegraphics[width=0.5\linewidth]{ls/scratch-card.png}}
	\caption{Scratch-Card biljet\cite{randell_ryan_voting_technologies_and_trust}}
	\label{fig:ls:scratch-card}
\end{figure}

\npar Op het rechterdeel van het biljet is onderaan een kraslaag aangebracht. Bovenop deze laag is het unieke ID (RIN) van het biljet geprint, dat de kiezer later kan gebruiken om zijn stem op het bulletin board terug te controleren. Bovendien verbergt deze laag een vooraf geprinte code die de volgorde van de kandidaten aangeeft (OCN). Bij het tellen wordt deze laag verwijderd en tegelijk verdwijnt ook het RIN van het biljet. Het is zeer belangrijk dat de RIN en OCN volledig ongecorreleerd zijn, want anders zou achterhaald kunnen worden van wie de stem is.

\npar Omdat het ticket een kopie is van het biljet met het kraslaagje nog intact, kan de OCN nooit meer achterhaald worden. Het is dus onmogelijk om uit het ticket af te leiden voor wie er gestemd is. Het is dus belangrijk goed te controleren dat de tellers niet proberen om RIN/OCN-combinaties neer te schrijven tijdens het verwijderen van het laagje.

\npar Een mogelijke toevoeging aan het systeem is om twee kopie\"en te maken van het biljet. De eerste wordt meegegeven met de kiezer als ticket, terwijl de tweede bewaard wordt in een doos. Onafhankelijke auditeurs controleren dan of een aantal willekeurige tickets correct op het bulletin board geplaatst zijn. Op die manier krijgen we een VVPAT-systeem waarbij de controles van de kiezers een toevoeging zijn aan deze van de auditeurs.

\npar Een nadeel aan het voorgestelde systeem is dat iedereen die een origineel rechterdeel bemachtigt, kan achterhalen op wie dat biljet gestemd heeft. Er is gelukkig wel geen rechtstreekse link tussen de RIN en de persoon die gestemd heeft. Een alternatief systeem print daarom een ID op het linker- en rechterdeel (CIN). Op het rechterdeel zit deze CIN opnieuw onder een kraslaag. In deze variant moet de kiezer na het stemmen ook zijn linkerdeel in een doos deponeren. Als ticket krijgt hij opnieuw een kopie mee van het rechterkant, waarop de kraslaag nog intact was.

\npar Om de stemmen te tellen, wordt opnieuw de kraslaag verwijderd zodat de CIN gelezen kan worden. Vervolgens moet de linkerkant met dezelfde CIN gevonden worden om te achterhalen op wie gestemd is. Het voordeel is dat het nu veel moeilijker is om stemmen toe te voegen of te verliezen omdat er twee kanten zijn. Daarnaast kan de privacy van de kiezer niet geschonden worden wanneer iemand alleen een rechterkant bemachtigd, aangezien deze geen informatie over de volgorde van de kiezers bevat. Het grootste probleem is dat het zoeken naar de juiste paren heel veel werk zou vragen bij grote verkiezingen, tenzij dit geautomatiseerd zou worden.

\section{Systemen met cryptografie}

De systemen in de vorige sectie maakten geen gebruik van ingewikkelde cryptografische technieken. Omdat het hierdoor eenvoudiger te begrijpen is, zal zo'n systeem sneller vertrouwd worden door de kiezer (\ref{sec:ls:vertrouwen}). In deze sectie worden toch enkele systemen besproken die wel steunen op cryptografie. Door gebruik te maken van zero-knowledge bewijzen, homomorfe cryptografie en mixnets kunnen immers veilige end-to-end verifiable systemen ontworpen worden.

\npar Bij Secret-Ballot Receipts (\ref{sec:ls:secret_ballot_receipts}) wordt optische cryptografie toegepast om de stem op het ticket te encrypteren. Scratch \& Vote (\ref{sec:ls:scratch_and_vote}) bouwt verder op de principes die geïntroduceerd werden bij Scratch-Card (\ref{sec:ls:scratch_card}).

\subsection{Secret-Ballot Receipts}
\label{sec:ls:secret_ballot_receipts}

Secret-Ballot Receipts\cite{chaum_secret_ballot} werd in 2004 gepubliceerd door David Chaum. De kiezer ziet zijn stem geprint worden in het stemhokje en kan zijn ticket gebruiken om nadien te controleren of ze correct meegeteld is. Omdat zijn keuzes ge\"encrypteerd worden tijdens het printproces kan hij het ticket niet gebruiken om te bewijzen hoe hij gestemd heeft. Bovendien is het niet nodig om trusted hardware te gebruiken aangezien gepubliceerde code kan op relatief eenvoudige systemen gedraaid worden.

\npar Nadat de kiezer zijn keuzes aangegeven heeft, worden deze door een speciale printer afgedrukt. De printer drukt tegelijk op beide kanten van het strookje afzonderlijke, maar uitgelijnde afbeeldingen. De kiezer wordt gevraagd om de afdruk te controleren en kan zijn stem eventueel nog aanpassen. Waneer hij tevreden is, kan hij kiezen of hij de boven- of onderkant wil meenemen. Pas dan wordt het laatste stukje van het ticket afgedrukt en kan hij de twee delen uit de printer nemen, terwijl ze nog aan elkaar vastzitten.

\npar Door de twee kanten van elkaar los te maken, wordt de afbeelding op het strookje een schijnbaar willekeurige afbeelding. Het doorgelaten licht op de plaatsen waar geen van beide kanten bedrukt is, maakte de stem zichtbaar. Geen van beide lagen bevat dus informatie over hoe gestemd is. Het laatst geprinte stukje is verschillend omdat daar wel tekst opstaat die ook na het scheiden van de twee lagen nog gelezen kan worden. Op de ene kant wordt duidelijk aangegeven dat deze bijgehouden moet worden en op de andere dat hij afgegeven moet worden. Deze laatste wordt duidelijk zichtbaar voor de kiezer vernietigd.

\begin{figure}[H]
	\center{\includegraphics[width=0.5\linewidth]{ls/secret-ballot_vote.png}}
	\caption{Strookje met optische ge\"encrypteerde stem\cite{chaum_secret_ballot}}
	\label{fig:ls:secret_ballot_vote}
\end{figure}

\begin{figure}[H]
	\center{\includegraphics[width=0.5\linewidth]{ls/secret-ballot_receipt.png}}
	\caption{Laatste stukje ticket met beide kanten nog samen\cite{chaum_secret_ballot}}
	\label{fig:ls:secret_ballot_receipt}
\end{figure}

\npar De computer houdt zelf een digitale versie van het volledige ticket bij en verwijdert ook de data van de andere kant. Deze data wordt na het aflopen van de stemming ge\"upload naar een online bulletin board. Omdat het ticket geen informatie bevat over de stem van de kiezer, kan hij dit aan iedereen tonen zonder zijn stem openbaar te maken. Door het ticket te scannen kan eenvoudig vastgesteld worden of het authentiek is. Bij een ongeldige controle is men dus zeker dat de apparatuur niet correct gewerkt heeft.

\npar De kiezer kan na de stemming nagaan of zijn ticket correct op het bulletin board staat. Hij kan dit eenvoudig doen door te kijken of de versie die daar staat volledig overeenkomt met zijn eigen ticket. Na het afsluiten van de stemming wordt de uiteindelijke verzameling van stemmen die geteld moeten worden online gezet. Er worden ook digitale handtekeningen van de set gepubliceerd die gebruikt kunnen worden om de echtheid ervan te controleren. Wanneer de stemmen geteld zijn, wordt een nieuwe set online geplaatst. Deze bevat even veel biljetten, maar nu zijn afbeeldingen gedecrypteerd en is elke stem leesbaar. Om de privacy van de kiezer te bewaren, zijn de biljetten willekeurig geordend.

\npar Er wordt gebruikt gemaakt van een audit proces om te controleren of beide sets identiek dezelfde biljetten bevatten. Het telproces verloopt in verschillende stappen\cite{chaum_secret_ballot} en na elke stap wordt een klein aantal willekeurig gekozen biljetten gedecrypteerd van de set tussen twee stappen in het telproces. Deze biljetten worden zo gekozen dat ze niet voldoende informatie bevatten zodat een kiezer ge\"identificeerd kan worden, maar wel gebruikt kunnen worden om na te gaan of er geen biljetten toegevoegd, verwijderd of gewijzigd werden.

\npar Omdat de optische encryptie neerkomt op een one-time pad, kan zelfs een aanvaller met ongelimiteerde rekenkracht de stem niet achterhalen. Het is immers onmogelijk om de oorspronkelijke afbeelding te achterhalen tenzij je de twee kanten van het biljet hebt. De gebruikte sleutels zijn dus de pixels van één van beide kanten. Deze zijn niet willekeurig, maar in de praktijk kunnen ze hiervan niet onderscheiden worden tenzij door de personen die de decryptie zullen uitvoeren. Het is technisch mogelijk het volledige biljet de juiste stemmen te laten tonen, maar op \'e\'en van de kanten toch een andere te encoderen. Dit kan alleen wanneer \'e\'en van de kanten ongeldig is, want het zou zeker opgemerkt worden wanneer beiden foutief zijn. Daarom is het belangrijk dat de kiezer zelf kan kiezen welke kant hij meeneemt als ticket. Zo is er immers een kans van 50\% dat een ongeldige kant gedecteerd zou worden.

\subsection{Scratch \& Vote}
\label{sec:ls:scratch_and_vote}

Scratch \& Vote\cite{adida_rivest_scratch_and_vote} werd in 2006 ontworpen door Ben Adida en Ronald L. Rivest. Het is een variatie op Scratch-Card (\ref{sec:ls:scratch_card}) waarbij gebruik gemaakt wordt van homomorfe cryptografie en zero-knowledge correctheidsbewijzen. Iedereen kan het uiteindelijke resultaat verifi\"eren en alleen de cijfertekst van de uitslag moet gedecrypteerd worden door de verantwoordelijken van de verkiezing. Het grote verschil met Scratch-Card is dat er niet langer met een RIN/CIN gewerkt moet wordt, net omdat de stemmen nu ge\"encrypteerd worden.

\npar Bij het aanmelden ontvangt de kiezer een biljet dat uit twee delen bestaat. Op de linkerzijde staan de kandidaten in een willekeurige volgorde, die alleen door de kiezer gezien mag worden. Op de rechterkant kan de kiezer zijn stem uitbrengen. Onderaan dit deel staan verder een 2D barcode en een kraslaag. Net zoals bij Scratch-Card wordt de linkerkant na het invullen van het biljet in het stemhokje afgescheurd en in een doos gedeponeerd. Een bijzitter controleert of de kraslaag op de rechterkant nog intact is en verwijdert deze daarna. Vervolgens wordt dit stukje zichtbaar voor de kiezer vernietigd. Tot slot laat de kiezer de eigenlijke stem en barcode scannen. Wanneer het stukje met de kraslaag verwijderd is van het biljet, bevat het rechterdeel geen informatie meer die gebruikt kan worden om de stem van de kiezer te achterhalen. Het gescande deel kan dus als ticket meegenomen worden.

\npar Door aan te melden op het bulletin board, kan de kiezer controleren of zijn biljet correct gescand werd. Wanneer dit niet het geval is, kan hij met zijn ticket gaan klagen. Omdat alle gescande biljetten online geplaatst worden, kan iedereen nagaan dat de cijfertekst van de eindtelling correct is.

\begin{figure}[H]
	\center{\includegraphics[width=0.5\linewidth]{ls/scratch_and_vote_ballot.png}}
	\caption{Scratch \& Vote biljet\cite{adida_rivest_scratch_and_vote}}
	\label{fig:ls:scratch_and_vote_ballot}
\end{figure}

\npar Voor de encryptie wordt gebruik gemaakt van het Paillier public key cryptosystem. Dit systeem heeft een additief homomorfisme door het vermenigvuldigen van de cijferteksten. Het tellen van de stemmen bij een verkiezing met meerdere kandidaten zou echter niet mogelijk zijn zonder gebruik te maken van een multi-counter. Het aantal beschikbare bits voor de leesbare tekst onderverdeeld in verschillende tellers. Hierbij wordt ervoor gezorgd dat er voldoende bits beschikbaar voor elke teller zodat ze niet in elkaar kunnen overlopen.

%TODO Correct?
\npar Het systeem maakt daarnaast gebruik van zero-knowledge bewijzen. Deze worden gebruikt om aan te tonen dat een set cijferteksten $c_1, c_2, \ldots, c_l$ de encryptie is van de permutatie van $m_1, m_2, \ldots, m_k$ (ervan uitgaande dat geen twee subsets van ${m_i}$ dezelfde som hebben.\cite{adida_rivest_scratch_and_vote}

\npar Daarnaast worden ook bewijzen opgesteld die aantonen dat de biljetten zelf correct zijn. Omdat deze bewijzen te lang zijn om op de biljetten te printen, worden ze voor de start van de verkiezing ge\"upload naar het bulletin board. Ze worden tijdens het tellen van de stemmen gebruikt om te verzekeren dat elk biljet maar \'e\'en stem uitbrengt per verkiezing. Om de kiezer te garanderen dat zijn biljet tijdens het tellen niet ongeldig verklaard zal worden, wordt ook een offici\"ele lijst van alle geldige biljetten voorzien. De kiezer kan dan eenvoudig nagaan of zijn biljet hierop voorkomt.

\begin{figure}[H]
	\center{\includegraphics[width=0.5\linewidth]{ls/scratch_and_vote_encryption.png}}
	\caption{Scratch \& Vote encryptie\cite{adida_rivest_scratch_and_vote}}
	\label{fig:ls:scratch_and_vote_encryption}
\end{figure}

\npar De 2D barcode op elk biljet encodeert de willekeurige volgorde van de cijferteksten voor de verschillende kandidaten samen met een hash van de publieke sleutel (\ref{fig:ls:scratch_and_vote_encryption}). De startwaarde van de teller van elke kandidaat wordt samen met een willekeurige waarde ge\"encrypteerd. Zo heeft elk biljet een unieke cijfertekst voor elke kandidaat. Deze willekeurige waarden worden verborgen door de kraslaag. De startwaarden van de tellers vormen samen met de publieke sleutel de gepubliceerde parameters. Samen met de willekeurige waarden kunnen ze dus gebruikt worden om de volgorde van de kandidaten op het biljet te achterhalen. Daarom is het belangrijk dat dit stukje van het biljet vernietigd wordt.

\npar Deze informatie wordt toch op de biljetten geprint omdat ze nodig zijn voor een audit van de biljetten. Dit wordt gedaan door de kiezer twee biljetten te laten kiezen. Door de kraslaag weg te halen, worden de willekeurige waarden zichtbaar en kan nagegaan worden of het biljet correct is. Door het verwijderen van de kraslaag wordt het biljet ongeldig, wat de reden is dat de kiezer twee biljetten moet nemen. Op deze manier wordt de helft van de biljetten getest en dus is de kans groot dat foutieve biljetten gedetecteerd worden.


  \backmatter

  % Bibliography
  \nocite{*}

  \bibliographystyle{plain}
  \bibliography{references}

\end{document}
