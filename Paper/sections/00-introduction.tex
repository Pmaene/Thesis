% 
% Introduction
% @author Pieter Maene <pieter.maene@student.kuleuven.be>
%

\section{Introduction}

Elections are an important part of our democratic society. In 2014, national elections will be held in 40 countries and 42\% of the world population will be able to cast its vote.\cite{news:economist_2014_ballot_boxes} It is therefore necessary to have a trustworthy voting system. In current systems, voters usually have to trust the election organiser. Voter-verifiable systems allow them to verify whether their ballot was registered correctly and check the election result. Helios is such a voter-verifiable system, that can be used for online elections. It is an open-source project developed by Ben Adida.\cite{site:adida_helios_documentation} Online elections can only be used in situations where coercion is not a big threat, though. Robbert Coeckelbergh added threshold encryption to it, so that one trustee can no longer stall the election.\cite{coeckelbergh_application_and_extension_of_the_helios_voting_system} This paper will discuss the procedure that should be followed to create an election in Helios. The interface of the system was also modified to better support this procedure. Finally, the system was applied in a real election.

\par The Web Cryptography API is a new W3C specification that will add cryptographic functionality to user agents.\cite{sleevi_watson_web_cryptography_api} The performance of this new API will be compared to that of existing JavaScript cryptography libraries. If found better, it could improve the usability of web applications that need user-level cryptography.
