%
% Conclusion
% @author Pieter Maene <pieter.maene@student.kuleuven.be>
%

\section{Conclusion}

Voter-verifiable systems can be used to improve voter confidence. Helios is such a system that can be used to organise online elections. \ref{sec:procedure} gave the procedure as well as some recommendations that should be followed to setup an election.

\par The Web Cryptography API will make high-level cryptographic functions available to web developers. It also offers a considerable performance increase, compared to existing JavaScript libraries. Its biggest disadvantage is the small set of supported algorithms, which limits the possible applications. It also proved hard to import existing keys into the user agent's store.

\par By modifying Helios' interface to better support the procedure, its usability was greatly improved. Where the system was hard to use before, the election administrator didn't need any support in setting up the election. The trustees were also able to manage their keys without any problems. Using the simplified voting booth, a typical number of voters cast their vote in a real election.